% Autogenerated translation of 2017-04-29-one-year.markdown by Texpad
% To stop this file being overwritten during the typeset process, please move or remove this header

\documentclass[12pt]{book}
\usepackage{graphicx}
\usepackage[utf8]{inputenc}
\usepackage[a4paper,left=.5in,right=.5in,top=.3in,bottom=0.3in]{geometry}
\setlength\parindent{0pt}
\setlength{\parskip}{\baselineskip}
\renewcommand*\familydefault{\sfdefault}
\usepackage{hyperref}
\pagestyle{plain}
\begin{document}
\Large

\hrule
layout: post
title:  "Reflections on One Year of Grad School"
date:   2017-04-29 09:00:00 -0000
categories: life

\section*{group: life}

As of March 28th, 2017, I've been a graduate student for one year at
the Université catholique de Louvain, funded by a fellowship from the
Erasmus Mundus Joint Doctorate program in the European Union.  This post chronicles my journey of obtaining a visa, moving to Belgium, and the experiences from the first year of my Ph.D.

The expectation was that I would quit my job at Basho in August 2015, which I did, take some short-term contracting assignments while preparing to move, obtaining the required paperwork, and generally wrapping up my life in the United States before setting off.  This didn't go exactly according to plan, because of complicated circumstances around obtaining a visa.

\section*{Moving to San Francisco}

I took a job at a company in San Francisco as a temporary assignment --
they knew I would only be there for few months before I obtained my visa
for Belgium -- and subsequently moved immediately to San Francisco from
Paris, where I had been living for a few months at the end of my time at
Basho, given they were a remote company.  With a quick stop over in Providence, I sold and donated all of my belongings (to various charitable organizations, friends, etc.) which included a very large vinyl record collection, four turntables and recording equipment, over 1,500 novels and literature, a bunch of vintage computers including a several SparcStations, a SGI Indy, and my original Commodore 64, and my car.  It was sad to see it all go, but it was time to start a new life; a new life abroad!

I was both excited, and naive.

Given I was hoping to leave San Francisco after a few months, I arranged to stay in an
Airbnb in the Tenderloin when I arrived, and then find an apartment that
would house me until I was able to get a visa.  This ended up being a
much longer process than originally imagined.

Once I arrived, I soon began working on visa.  I'd soon come to learn
that I didn't need an entry visa into Belgium because I'm a US citizen,
but I began the process of getting the visa as instructed by the
university.

\section*{Getting the visa}

Getting the visa ended up being a complicated process.  I arranged to
see a doctor in SF on Russian Hill, who does immigration exams and went
through a rudimentary examination.  The form he was instructed to fill
out, that I printed from the Belgian consulates website, looked so fake
that he didn't believe the form was real at first.  He then went to the
website himself and found out that the form was, in fact, a real form
and proceeded to fill it out and gave it to me.  I took this form to a
notary in SF to get it notarized, and they didn't accept it; they needed
the doctor to sign it.  I told the doctor this and then arranged to have
the form notarized; they scheduled an appointment and then visited the
notary.

This also wasn't enough.  I later found out that I needed the form
designated from an international notary (keep in mind, this form states
that I didn't have tuberculosis) and I'd have to repeat the process.  I
took the form to an international notary, who was mobile and wasn't
there when I arrived in the office, and later had to schedule an
appointment.  I was then informed that this form needed to go to
Sacramento, that's the only place where the officiant was, and I'd have
to supply an overnight postage prepaid to get it the next day.

[So far, we've spend \$500, given FBI fingerprints, overnight background
checks, and a medical report that wasn't covered under insurance.]

I finally received this information after a few days and then set off to
schedule a single day flight to Los Angeles so I could visit the Belgian
consulate and retrieve my visa.

Meanwhile, I've had to try to get an apartment in SF for a temporary
stay, and before I found a friend to put me up, I ended up staying at an
Airbnb at 24th and Mission at a whopping \$2,200 a month for a private
room; I couldn't find a cheaper place because I couldn't commit to more
than week-by-week given my awaiting-on-a-visa status.

I make it to LA, head to the consulate, and have the forms signed provide
the information needed for them to get me the visa back to SF via
overnight mail, and receive my visa two days later.

[Total expenditure on the visa: \$1,500]

Once I received the visa, I receive an email from my advisor stating
that if I'm not in Belgium to sign the forms by March 28th, I forfeit my
fellowship.  I'm about to book a trip to Belgium, waiting for a response
from my advisor that states I can only be reimbursed for the travel to
Belgium and not prepaid, even though I had a travel allowance, when the
Brussels airport is attacked by terrorists.  I have three days to book a
ticket to de Gaulle, travel to Belgium by train, and make it there with
one day to spare.  I arrive and find out I don't have to physically sign
anything after all, and check into my university flat.

Meanwhile, I had quit my short-term job and was told that given I didn't
stay a year, I would forfeit all of my stock options and have to repay my
relocation benefits, even though they knew before I started I wouldn't
be there for more than a year because I was going to grad school; so
much for reading the contracts in depth: apparently I'm not lucid
enough.

[More on this another time: my job completely changed 4 days after I
started; and I went from researcher to software engineer.]

\section*{The First Year}

My first year at the university was decent.  Let's look at what happened
during this year.

\begin{itemize}
\item I spent over 200 days at affiliated universities and conferences, and
wasn't at my home university; this was nice, because I'm living in a
shared dorm with a shared bathroom and private shower;
\item I attended EuroSys, ICFP, ECOOP, and various other academic events;
\item I assisted in organizing Curry On;
\item I started the PMLDC workshop at ECOOP on distributed programming;
\item I co-chaired the Code Mesh conference;
\item I visited Cambridge, Braga, Oxford, IMDEA, and other universities on
research visits;
\item I spoke at Percona, Erlang User Conference, Erlang Factory, CRAFT
Conference, numerous Papers We Love meetups, GOTO London, GOTO
Chicago;
\item I was a subreviewer for CloudCom, and shadow PC for EuroSys;
\item I TA'd two courses; one on Cloud Computing and one on Distributed
Algorithms;
\item I mentored two students in the Google Summer of Code, one of which
presented a paper based on his work at AGERE! 2016;
\item I performed a large scale evaluation of Lasp for the SyncFree project
and ran a live demo of our research prototype at the final review at
the European Commission;
\item I scaled Lasp to 1024+ nodes, challenged the design of distributed
Erlang and showed higher scalability in Erlang than ever demonstrated
before;
\item I was the recipient of a Microsoft Research internship in Redmond this
summer;
\item Our Lasp prototype is used by one company and the supporting
infrastructure is used by another.
\end{itemize}

It's been a whirlwind year, but I'm still feeling stressed out, overwhelmed, and upset.  Why is that?

\begin{itemize}
\item I haven't published enough papers; we tried to publish a few things
related to our work but spend the better part of the year doing an
evaluation for the EU that totaled at over 9,900+ euros;
\item Erasmus does not provide enough money or time for student.  The salary
is good, but the costs involved in obtaining visa's, moving and
relocating every year is a pain;
\item Relocating every year is counterproductive, as much as it's supposed
to facilitate and support integration in the European Union;
\item I'm the only person in my lab, I sit alone and work alone.  This is
why I need to travel to do anything collaborative with my coworkers.
\item Moving takes a toll; it distracts from research, and cuts into your
time;
\item Is three years enough for a Ph.D.?  The funding that is provided only covers three years, but it seems that most of the students in my lab have needed at least a fourth, and have had to find additional funding for it.
\end{itemize}

Now, those are all academic complaints, but let's talk about more
practical complaints as an international student in Europe:

\begin{itemize}
\item I haven't been able to defer my loans from my undergraduate studies
yet.  This is a problem because the payment is over 1/2 of my rent as
a Ph.D. student;
\item Required travel and expenses have left me, at times, with over 3,000
euros of outstanding expenses.  It has taken the university over 12
months to reimburse me for these expenses;
\item Having to move every year means that you need to find an apartment
every year.  I've had problems where Portugal landlords haven't
responded, even yet, to my requests, and I haven't been able to secure
my apartment, which is required to get a visa;
\item Expedited passport renewals in the US are expensive -- I've paid over
\$400 for this -- and required for me to maintain my student status;
\item Belgian income, even if tax-free, is taxed in the United States, if
you are a citizen: the double taxation agreement only works on taxes
paid -- if you're tax free, you don't pay taxes and therefore don't
get any benefits;
\item Working alone and moving from lab to lab sucks.  It's impossible to
have a community and feel like you have a support system when you move
every year.  While the European Commission might think you are
promoting integration by moving people from country to country, what
it is not doing is making it so people can have good working
relationships and be healthy.
\end{itemize}

My time has been wonderful, but I sometimes wonder what it would be like
if the environment was more conducive to ensuring, first-class, that
Ph.D. students had the time and resources to perform good research
without feeling like they were running against the clock.

\end{document}
